\newcommand{\Amen}{\textsc{Amen.}}
\newcommand{\AAAmen}{\Amen{ }\Amen{ }\Amen}

\definecolor{myred}{rgb}{1,.1,0}
\definecolor{mygrey}{rgb}{.2,.2,.2}

\ifxetex
    \usepackage{soul}
    \newcommand{\titel}[1]{%
    	\phantomsection %für hyperref -- meckert sonst über addcontentsline
	\textcolor{myred}{\so{\textsc{#1}}}%
	\addcontentsline{toc}{section}{#1}
    }
    \newcommand{\Titel}[1]{%Titel, der gleichzeitig in die Liedliste geht
    	\phantomsection %für hyperref -- meckert sonst über addcontentsline
	\textcolor{myred}{\so{\textsc{#1}}}%
	\addcontentsline{toc}{section}{#1}
	\addcontentsline{lof}{figure}{#1}
    }
    \newcommand{\sperren}[1]{{\footnotesize\so{#1}}}
\else
    \newcommand{\titel}[1]{\textcolor{myred}{\textls[200]{\textsc{#1}}}}
    \newcommand{\sperren}[1]{{\footnotesize\textls[100]{#1}}}
\fi

\newcommand{\Liedliste}[1]{%zur Liedliste hinzufügen
    \addcontentsline{lof}{figure}{#1}
}

\newcommand{\kreuz}{\maltese\ } % symbol für Kreuz im Text 

\newcommand{\f}[1]{\textbf{#1}}

\newcommand{\trenner}[1]{% Trennerzeichen für verschiedene Optionen
	\begin{center}
	    $\blacklozenge$ \emph{#1}%\Diamond$
	\end{center}
	}

\newcommand{\doppeltrenner}[1][]{% Trennerzeichen für verschiedene Optionen
	\begin{center}
	    $\blacklozenge$ \emph{#1} $\blacklozenge$ %\Diamond$
	\end{center}
	}

\newcommand{\lit}[1]{% Liturg
	{\textsc{Liturg:}} #1
}

\newcommand{\gem}[1]{% gemeinsam
    \begin{addmargin}[1.5em]{.5em} #1 \end{addmargin}
}

\renewcommand{\Gem}[1]{% Gemeinde renew wegen myTeX
    {\footnotesize\begin{addmargin}[1.5em]{.5em}\emph{Gemeinde: }#1 \end{addmargin}}
}

\newcommand{\Antwort}[1]{% Antwort
    {\footnotesize\begin{addmargin}[1.5em]{.5em}\emph{Antwort: }#1 \end{addmargin}}
}


%braucht xparse!
\usepackage{xstring,xifthen}
%\Lied[<++Verse++>][<++Titel++>][<++Liederbuch++>]{<++Nummer++>}
\DeclareDocumentCommand \Lied { O{} O{} O{EG} m }{%Lied
\begin{addmargin}[1.5em]{.5em}%
    \StrLen{#1}[\VersLen]%
    \StrLen{#2}[\TitelLen]%
    \ifthenelse{\VersLen = 0}{% Kein Vers angegeben
    	\phantomsection
	\citetitle{#3:#4} \hfill{} (#3~#4)%,#1)%
	\addcontentsline{toc}{subsection}{#3~#4}%
	\addcontentsline{lof}{figure}{#3~#4}%
    }{%else
    \ifthenelse{\TitelLen = 0}{% Kein Titel angegeben
    	\phantomsection
	\citetitle{#3:#4} \hfill{} (#3~#4,#1)%
	\addcontentsline{toc}{subsection}{#3~#4,#1}%
	\addcontentsline{lof}{figure}{#3~#4,#1}%
    }{%else
    	\phantomsection
	#2 \hfill{} (#3~#4,#1)%
	\addcontentsline{toc}{subsection}{#2 -- #3~#4,#1}%
	\addcontentsline{lof}{figure}{#3~#4,#1}%
    }}
\end{addmargin}
}

\newcommand{\Regie}[1]{%
    	\phantomsection
	{\footnotesize \emph{#1}}
	\addcontentsline{toc}{subsection}{#1}
}

\makeatletter %Setzt den Pagestyle auf den ersten Seiten eines Chapters neu, damit auch dort fwlw angewandt wird.
\renewcommand\chapter{\if@openright\cleardoublepage\else\clearpage\fi
                    \thispagestyle{scrheadings}% original style: plain
                    \secdef\@chapter\@schapter}
\makeatother

\renewcaptionname{german}{\listfigurename}{Liedliste}

\newcommand{\Jünger}{\gender{Jüngerinnen()Jünger}}
\newcommand{\Jüngern}{\gender{Jüngerinnen()Jüngern}}
\newcommand{\Christen}{\gender{Christinnen()Christen}}
