%Diese Datei liegt in ~/texmf/tex/latex/myTeX/ und wird verlinkt, ja?!
\usepackage{xstring, xifthen, refcount}

%Abkürzungen mit Punkten versehen.
%Kontrolliert nur auf 2 und 3! Ansonsten wird fett gedruckt!
\newcommand{\abk}[1]{%
    \StrLen{#1}[\MyStrLen]%kann nicht in ifthenelse verwendet werden!
    \ifthenelse{\equal{\MyStrLen}{2}}%
    {\StrChar{#1}{1}.\,\StrChar{#1}{2}.}%
    {\ifthenelse{\equal{\MyStrLen}{3}}%
	{\StrChar{#1}{1}.\,\StrChar{#1}{2}.\,\StrChar{#1}{3}.}%
	{\textbf{#1}}%
    }%
}
%
%--------------------
%
%Wenn das Label mit sec: beginnt, wird auf einen Abschnitt verwiesen, sonst auf eine Seite. benötigt substr
\newcommand{\siehe}[1]{%
    \IfSubStr{#1}{sec:}%
	{siehe Abschnitt \ref{#1}.}%
	{\ifthenelse{\getpagerefnumber{#1}>\thepage}{s.\,u.~S.\,\pageref{#1}.}{s.\,o.~S.\,\pageref{#1}.}}
}

\newcommand{\Siehe}[1]{%
    \IfSubStr{#1}{sec:}%
	{Siehe Abschnitt \ref{#1}.}%
	{\ifthenelse{\getpagerefnumber{#1}>\thepage}{S.\,u.~S.\,\pageref{#1}.}{S.\,o.~S.\,\pageref{#1}.}}
}
%
%--------------------
%
\newcommand{\versnr}[1]{\raisebox{1ex}{{\tiny#1}}\kern -.125em}

\newsavebox{\boxbibelstelle}
\newenvironment{bibelzitat}[1][]
     {\small\itshape\sbox{\boxbibelstelle}{#1}
     \begin{quote}\singlespacing}
     {\hspace*{\fill}\usebox{\boxbibelstelle}\end{quote}}

\newsavebox{\boxlangzitat}
\newenvironment{langzitat}[1][]
     {\small%\itshape
     \sbox{\boxlangzitat}{#1}
     \vspace{-4ex}%Teständerung!
     \begin{quotation}\singlespacing}%\vspace{-3ex}}%neg. vspace, da sonst das Zitat zu weit unten beginnt. Zickt!
     {\hspace*{\fill}\usebox{\boxlangzitat}\end{quotation}
     \vspace{-1ex}
     }

\newcommand{\bibel}{\pbibleverse} %aus bibleref-parse -> \bibel{Joh 3,16}
%
%--------------------
%
%Silbentrennung auf Seitenwechsel verbieten
%\brokenpenalty=1000
% Keine "Schusterjungen"
\clubpenalty = 10000
% Keine "Hurenkinder"
\widowpenalty = 10000
\displaywidowpenalty = 10000
%
%--------------------
%
\ifxetex %XeLaTeX--------------------
    \newcommand{\lat}[1]{\emph{\textlatin{#1}}\hspace{.05em}}
    \newcommand{\zit}[1]{\emph{\enquote{#1}}}
    \newcommand{\zn}[1]{\enquote{#1}}
    \newcommand{\znn}[1]{\enquote*{#1}}
    %\newcommand{\zit}[1]{{\textit{"`#1"'}\hspace{.1em}}} %dt Anführungszeichen und italic mit kleinem Zwischenraum danach
    %\newcommand{\zn}[1]{"`#1"'} %dt Anführungszeichen
    %\newcommand{\znn}[1]{‚{#1}‘} %dt einfache Anführungszeichen
\else %pdfLaTeX--------------------
    \newcommand{\lat}[2][ngerman]{\foreignlanguage{#1}{\emph{#2}\hspace{.05em}}}
    \newcommand{\zit}[1]{{\textit{\glqq#1\grqq}\hspace{.1em}}} %dt Anführungszeichen und italic mit kleinem Zwischenraum danach
    \newcommand{\zn}[2][ngerman]{\glqq{\foreignlanguage{#1}{#2}}\grqq} %dt Anführungszeichen
    \newcommand{\znn}[2][ngerman]{\glq{\foreignlanguage{#1}{#2}}\grq} %dt einfache Anführungszeichen
\fi
%
%--------------------
%
\newcommand{\name}[1]{\textsc{#1}}

\newcommand{\URL}[2]{%
{\textsc{url:} \url{#1} (abgerufen am #2)}
}
%
%--------------------
%
\newcommand{\RU}{R\kern-.07emU}%Religionsunterricht}
\newcommand{\SuS}{S\kern-.05emu\kern-.05emS}

\newcommand{\GD}{Gottesdienst}
\newcommand{\AM}{Abendmahl}
\newcommand{\KU}{Konfirmandenunterricht}
\newcommand{\KA}{Konfirmandenarbeit}
\newcommand{\Ko}{Konfirmation}
\newcommand{\Konfis}{\gender{Konfirmandinnen()Konfirmanden}}

\newcommand{\Jünger}{\gender{Jüngerinnen()Jünger}}
\newcommand{\Jüngern}{\gender{Jüngerinnen()Jüngern}}
\newcommand{\Christen}{\gender{Christinnen()Christen}}

\newcommand{\HERR}{\name{Herr}}
\newcommand{\HERRN}{\name{Herrn}}

\newcommand{\rkK}[1][]{rö"-misch"=ka"-tho"-lische#1 Kirche}

\newcommand{\Gem}{Gemeinde}

\newcommand{\JC}{Jesus Christus}
\newcommand{\Pls}{Paulus}

\newcommand{\HKST}{Hei"-lig"-kreuz"-stei"-nach}
\newcommand{\BB}{Brom"-bach}
\newcommand{\HB}{Hed"-des"-bach}
\newcommand{\HD}{Hei"-del"-berg}

\newcommand{\BW}{Ba"-den"=Württem"-berg}

\newcommand{\Lra}{$\Longrightarrow$}
\newcommand{\Ra}{$\Rightarrow$}

\newcommand{\jedejeder}{\gender{jede()jeder}}
\newcommand{\jedejeden}{\gender{jede()jeden}}
\newcommand{\jederjedes}{\gender{jeder()jedes}}
\newcommand{\jederjedem}{\gender{jeder()jedem}}

%
%--------------------
%
%Fußnotenanpassung
\deffootnote{2.5em}% Einzug des Fußnotentextes samt Fußnotenziffer
{1em}% zusätzlicher Absatzeinzug in der Fußnote
{\makebox[2em]% Raum für Fußnotenzeichen: ebenso groß wie Einzug des FN-Textes
[r]% Ausrichtung des Fußnotenzeichens: [r]echts, [l]inks
{\thefootnotemark}%
\hspace{1em}% Abstand zw. FN-Zeichen und FN-Text}%
}
%
%--------------------
%
\newcommand{\pfeil}[1]{% ACHTUNG: kein Fehler angezeigt, bei fehlerhafter Eingabe
\IfStrEq{#1}{->}{$\rightarrow$}{\relax}%
\IfStrEq{#1}{-->}{$\longrightarrow$}{\relax}%
\IfStrEq{#1}{=>}{$\Rightarrow$}{\relax}%
\IfStrEq{#1}{==>}{$\Longrightarrow$}{\relax}%
\IfStrEq{#1}{<-}{$\leftarrow$}{\relax}%
\IfStrEq{#1}{<--}{$\longleftarrow$}{\relax}%
\IfStrEq{#1}{<=}{$\Leftarrow$}{\relax}%
\IfStrEq{#1}{<==}{$\Leftarrow$}{\relax}%
\IfStrEq{#1}{<->}{$\leftrightarrow$}{\relax}%
\IfStrEq{#1}{<-->}{$\longleftrightarrow$}{\relax}%
\IfStrEq{#1}{<=>}{$\Leftrightarrow$}{\relax}%
\IfStrEq{#1}{<==>}{$\Longleftrightarrow$}{\relax}%
}
%
%--------------------
%

